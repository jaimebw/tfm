\chapter{Introduction}



\section{Document Structure}

Since a report, and especially a thesis, might be a substantial document, it is convenient to break it up into smaller pieces. In this template we therefore give every chapter its own file. The chapters (and appendices) are gathered together in \texttt{report.tex}, which is the master file describing the overall structure of the document. \texttt{report.tex} starts with the line
\begin{quote}
    \texttt{\textbackslash documentclass\{uem-report\}}
\end{quote}
which loads the UEM report template. The template is based on the \LaTeX{} \texttt{book} document class and stored in \texttt{uem-report.cls}. 
\begin{quote}
    \texttt{\textbackslash documentclass[dutch]\{uem-report\}}
\end{quote}
Furthermore, hyperlinks are shown in blue, which is convenient when reader the report on a computer, but can be expensive when printing. They can be turned black with the \texttt{print} option. This will also turn the headers black instead  of cyan.

If the document becomes large, it is easy to miss warnings about the layout in the \LaTeX{} output. In order to locate problem areas, add the \texttt{draft} option to the \texttt{\textbackslash documentclass} line. This will display a vertical bar in the margins next to the paragraphs that require attention. Finally, the \texttt{nativefonts} option can be used to override the automatic font selection (see below).

This template has the option to automatically generate a cover page with the \texttt{\textbackslash makecover} command. See the next section for a detailed description.

The contents of the report are included between the \texttt{\textbackslash begin\{document\}} and \texttt{\textbackslash end\{document\}} commands, and split into three parts by
\begin{enumerate}
\item\texttt{\textbackslash frontmatter}, which uses Roman numerals for the page numbers and is used for the title page and the table of contents;
\item\texttt{\textbackslash mainmatter}, which uses Arabic numerals for the page numbers and is the style for the chapters;
\item\texttt{\textbackslash appendix}, which uses letters for the chapter numbers, starting with `A'.
\end{enumerate}
The title page is defined in a separate file, \emph{e.g.}, \texttt{title.tex}, and included verbatim with \texttt{\textbackslash input\{title\}}.\footnote{Note that it is not necessary to specify the file extension.} Additionally, it is possible to include a preface, containing, for example, the acknowledgements. An example can be found in \texttt{preface.tex}. The table of contents is generated automatically with the \texttt{\textbackslash tableofcontents} command. Chapters are included after \texttt{\textbackslash mainmatter} and appendices after \texttt{\textbackslash appendix}. For example, \texttt{\textbackslash input\{chapter-1\}} includes \texttt{chapter-1.tex}, which contains this introduction.

The bibliography, finally, is generated automatically with
\begin{quote}
    \texttt{\textbackslash bibliography\{report\}}
\end{quote}
from \texttt{report.bib}. The bibliography style is specified in \texttt{uem-report.bst}, which is a modified version of \texttt{apsrev4-1.bst} (from REVTeX) designed to also display the titles of referenced articles. The template will automatically generate clickable hyperlinks if a URL or DOI (digital object identifier) is present for the reference. As an example, we cite the paper by Nobel laureate Andrei Geim and his pet hamster \cite{Geim2001}. Although it is possible to manage the bibliography by hand, we recommend using EndNote (available from Blackboard) or JabRef (available from \url{http://jabref.sourceforge.net/}).

\section{Cover and Title Page}

This template will automatically generate a cover page if you issue the \texttt{\textbackslash makecover} command. There are two formats for the cover page: one with a page-filling (`bleeding')
illustration, with the title(s) and author(s) in large ultrathin typeface, and the other where the illustration fills the lower half of the A4, whereas title(s), author(s) and additional
text are set in the standard sans-serif font on a plain background with a color chosen by the user. The last option is selected by the optional key \texttt{split}: \texttt{\textbackslash makecover[split]} yields
a page with the illustration on the lower half. All illustrations are bleeding, in accordance with the UEM style.

Before generating the cover, you need to provide the information to put on it. This can be done with the following commands:
\begin{itemize}
\item\texttt{\textbackslash title[Optional Color]\{Title\}} \\
    This command is used to provide the title of the document. The title
    title is also printed on the spine. If you use a title page (see below), this information will be used there as well.
    As the title, subtitle and author name are printed directly over the cover photo, it will often be necessary to adjust the print color in order to have
    sufficient contrast between the text and the background. The optional color argument is used for this.
\item\texttt{\textbackslash title[Optional Color]\{Subtitle\}} \\
    This command is used to provide a subtitle for the document. If you use a title page (see below), this information will be used there as well.
    It possible to adjust the print color in order to have
    sufficient contrast between the text and the background -- the optional color argument is used for this.
\item\texttt{\textbackslash author\{J.\ Random Author\}} \\
    This command specifies the author. The default color is \texttt{uem-white}, but this may be adjusted in the same way as the titles.
\item\texttt{\textbackslash affiliation\{Technische Universiteit Delft\}} \\
    The affiliation is the text printed vertically on the front cover. It can be the affiliation, such as the university or department name, or be used for the document type (\emph{e.g.}, Master's thesis). The default color is again \texttt{uem-white}, adjustable through the \texttt{color} option.
\item\texttt{\textbackslash coverimage\{cover.jpg\}} \\
    With this command you can specify the filename of the cover image. The image is stretched to fill the full width of the front cover (including the spine if a back cover is present).
\item\texttt{\textbackslash covertext\{Cover Text\}} \\
    If a back cover is present, the cover text is printed on the back. Internally, this text box is created using the \LaTeX{} \texttt{minipage} environment, so it supports line breaks.
\item\texttt{\textbackslash titleoffsetx\{OffsetX\},\textbackslash titleoffsety\{OffsetY\}}
    If the cover page contains a page-filling picture (i.e., \texttt{split} is not specified with the \texttt{makecover} command, the best position of the title depends a lot on the picture chosen for it. The lower left corner of the minipage containing title, subtitle and author is 
    specified by these two commands. The offsets are measured from the top left corner of the page. 
\item\texttt{\textbackslash afiloffsetx\{AfilX\}, \textbackslash afiloffsety\{AfilY\}}
    specifies the lower left corner of the text containing the affiliation, measured from the top left corner of the page. 
\end{itemize}

In addition to \texttt{[split]}, the \texttt{\textbackslash makecover} command accepts several additional options for customizing the layout of the cover. 
The most important of these is \texttt{back}. Supplying this option will generate a back cover as well as a front, including the spine. Since this requires a page size slightly larger than twice A4 (to make room for the spine), and \LaTeX{} does not support different page sizes within the same document, it is wise to create a separate file for the cover. \texttt{cover.tex} contains an example. The recommended page size for the full cover can be set with
\begin{quote}
    \textbackslash geometry\{papersize=\{1226bp,851bp\}\}
\end{quote}
after the document class and before \texttt{\textbackslash begin\{document\}}.

The other options \texttt{\textbackslash makecover} accepts are
\begin{itemize}
\item\texttt{nospine} \\
    If a back cover is generated, the title will also be printed in a black box on the spine. However, for smaller documents the spine might not be wide enough. Specifying this option disables printing the title on the spine.
\item\texttt{frontbottom} \\
    By default the black box on the front is situated above the blue box. Specifying this option will place the black box below the blue one.
\item\texttt{spinewidth} \\
    If a back cover is present, this option can be used to set the width of the spine. The default is \texttt{spinewidth=1cm}.
\item\texttt{frontboxwidth}, \texttt{frontboxheight}, \texttt{backboxwidth}, \texttt{backboxheight} \\
    As their names suggest, these options are used to set the width and height of the front (black) and back (blue) boxes. The default widths and heights are \texttt{4.375in} and \texttt{2.1875in}, respectively.
\item\texttt{x}, \texttt{y} \\
    The blue and black boxes touch each other in a corner. The location of this corner can be set with these options. It is defined with respect to the top left corner of the front cover. The default values are \texttt{x=0.8125in} and \texttt{y=3in}.
\item\texttt{margin} \\
    This option sets the margin between the borders of the boxes and their text. The default value is \texttt{12pt}.
\end{itemize}

For a thesis it is desirable to have a title page within the document, containing information like the thesis committee members. To give you greater flexibility over the layout of this page, it is not generated by a command like \texttt{\textbackslash makecover}, but instead described in the file \texttt{title.tex}. Modify this file according to your needs. The example text is in English, but Dutch translations are provided in the comments. Note that for a thesis, the title page is subject to requirements which differ by faculty. Make sure to check these requirements before printing.

\section{Chapters}

Each chapter has its own file. For example, the \LaTeX{} source of this chapter can be found in \texttt{chapter-1.tex}. A chapter starts with the command
\begin{quote}
    \texttt{\textbackslash chapter\{Chapter title\}}
\end{quote}
This starts a new page, prints the chapter number and title and adds a link in the table of contents. If the title is very long, it may be desirable to use a shorter version in the page headers and the table of contents. This can be achieved by specifying the short title in brackets:
\begin{quote}
    \texttt{\textbackslash chapter[Short title]\{Very long title with many words which could not possibly fit on one line\}}
\end{quote}
Unnumbered chapters, such as the preface, can be created with \texttt{\textbackslash chapter*\{Chapter title\}}. Such a chapter will not show up in the table of contents or in the page header. To create a table of contents entry anyway, add
\begin{quote}
    \texttt{\textbackslash addcontentsline\{toc\}\{chapter\}\{Chapter title\}}
\end{quote}
after the \texttt{\textbackslash chapter} command. To print the chapter title in the page header, add
\begin{quote}
    \texttt{\textbackslash setheader\{Chapter title\}}
\end{quote}

Chapters are subdivided into sections, subsections, subsubsections, and, optionally, paragraphs and subparagraphs. All can have a title, but only sections and subsections are numbered. As with chapters, the numbering can be turned off by using \texttt{\textbackslash section*\{\ldots\}} instead of \texttt{\textbackslash section\{\ldots\}}, and similarly for the subsection.
\section{\textbackslash section\{\ldots\}}
\subsection{\textbackslash subsection\{\ldots\}}
\subsubsection{\textbackslash subsubsection\{\ldots\}}
\paragraph{\textbackslash paragraph\{\ldots\}}



